The use of \acrfull{sicfet} sensors  in cyclic operation is a proven way to quantify different gases. Standard workflow involves extracting shape defining features such as average and slope of the sensor signal. This work's main goal is to verify if frequency modulation can be used to simultaneously quantify Nitric Oxide (\ch{NO}), Nitrogen Dioxide (\ch{NO2}) and Ammonia (\ch{NH3}). Linear models were chosen, namely: \acrfull{ols}, \acrfull{pcr}, \acrfull{plsr} and Ridge regression. Results indicate that these models fail to predict concentrations completely for every gas. Analysis indicates that the features are not linear in terms of concentrations. This work is concluded by recommending a few other alternatives before discarding frequency cycling completely: non-parametric models of regression and different frequency regime, namely the use of triangular waves in future experiments.