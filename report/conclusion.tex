%%% lorem.tex --- 
%% 
%% Filename: lorem.tex
%% Description: 
%% Author: Ola Leifler
%% Maintainer: 
%% Created: Wed Nov 10 09:59:23 2010 (CET)
%% Version: $Id$
%% Version: 
%% Last-Updated: Wed Nov 10 09:59:47 2010 (CET)
%%           By: Ola Leifler
%%     Update #: 2
%% URL: 
%% Keywords: 
%% Compatibility: 
%% 
%%%%%%%%%%%%%%%%%%%%%%%%%%%%%%%%%%%%%%%%%%%%%%%%%%%%%%%%%%%%%%%%%%%%%%
%% 
%%% Commentary: 
%% 
%% 
%% 
%%%%%%%%%%%%%%%%%%%%%%%%%%%%%%%%%%%%%%%%%%%%%%%%%%%%%%%%%%%%%%%%%%%%%%
%% 
%%% Change log:
%% 
%% 
%% RCS $Log$
%%%%%%%%%%%%%%%%%%%%%%%%%%%%%%%%%%%%%%%%%%%%%%%%%%%%%%%%%%%%%%%%%%%%%%
%% 
%%% Code:

\chapter{Conclusion}
\label{cha:conclusion}

Given previous discussions and analysis, the answer to the research question "Can frequency modulation be used to simultaneously quantify \nox and Ammonia concentrations?" seems to be: "perhaps not". Although poor predictions, the methods used here are far from exhausting the several other, possibly more flexible, models in the statisticians toolbox. Moreover, experiments using different frequency modulations (e.g. triangular waves instead of square) and/or different sensor calibrations and/or different temperatures could be further investigated in order to answer this question conclusively.

As for the second question "Which method yields best predictions of gas concentrations?", the correct answer would be "none". Nonetheless, out of all proposed methods, \acrshort{plsr} has best relative performance in terms of \acrshort{rmse}.

The author finds comfort in perhaps pointing future work towards better methods and possibly better quantification of these gases in hopes of addressing the problem more efficiently than current practices. In this sense, this thesis work is considered successful.

