%%% Intro.tex --- 
%% 
%% Filename: Intro.tex
%% Description: 
%% Author: Ola Leifler
%% Maintainer: 
%% Created: Thu Oct 14 12:54:47 2010 (CEST)
%% Version: $Id$
%% Version: 
%% Last-Updated: Thu May 19 14:12:31 2016 (+0200)
%%           By: Ola Leifler
%%     Update #: 5
%% URL: 
%% Keywords: 
%% Compatibility: 
%% 
%%%%%%%%%%%%%%%%%%%%%%%%%%%%%%%%%%%%%%%%%%%%%%%%%%%%%%%%%%%%%%%%%%%%%%
%% 
%%% Commentary: 
%% 
%% 
%% 
%%%%%%%%%%%%%%%%%%%%%%%%%%%%%%%%%%%%%%%%%%%%%%%%%%%%%%%%%%%%%%%%%%%%%%
%% 
%%% Change log:
%% 
%% 
%% RCS $Log$
%%%%%%%%%%%%%%%%%%%%%%%%%%%%%%%%%%%%%%%%%%%%%%%%%%%%%%%%%%%%%%%%%%%%%%s
%% 
%%% Code:

\chapter{Introduction}
\label{cha:introduction}

\section{Motivation}
\label{sec:motivation}

Nitric Oxide ({\ch{NO}) and Nitrogen Dioxide (\ch{NO2}), commonly referred together as \nox,  are hazardous gases to the environment and to humans. Its main sources are combustion processes in transportation, and industrial processes such as (but not limited to) auto mobiles, trucks, boats, industrial boilers, turbines, etc. \cite{EPA_2019}.

\nox exposure to humans can cause respiratory illnesses such bronchitis, emphysema and can worsen heart disease \cite{Boningari_2016}. Environmentally, \nox are deemed precursors of adverse phenomena such as smog, acid rain, and the depletion of ozone (\ch{O3}) \cite{Bernabeo_2019}. It is of high interest, therefore, to reduce \nox emissions.

One well studied and successful method of reducing emissions is \acrfull{scr}, which consists in the reduction of \nox by ammonia (\ch{NH3}) into nitrogen gas (\ch{N2}) and water (\ch{H2O}) \cite{Forzatti_2001}, both harmless components. The process is based in the following reactions \cite{Forzatti_2001}:
\begin{itemize}
	\item \ch{4 NH3 + 4 NO + O2 -> 4 N2 + 6 H2O}
	
	\item \ch{2 NH3 + NO + NO2 -> 2 N2 + 3 H2O}
	
	\item \ch{8 NH3 + 6 NO2 -> 7 N2 + 12 H2O}
\end{itemize}


One key element in these reactions, however, is the amount of ammonia dosed into the \acrshort{scr} systems. Ammonia itself is hazardous to humans, causing skin and respiratory irritation, among other illnesses \cite{ASTDRA_2004}. More importantly, ammonia is one of the main sources of nitrogen pollution and it has direct negative impact on biodiversity via nitrogen deposition in soil and water \cite{RAND_2018}. Hence it is also desired to keep ammonia emissions to a minimum. Too much ammonia in the \acrshort{scr} catalyst will guarantee \nox reduction at the expense of undesired ammonia emissions. Concurrently, too little ammonia will impede \acrshort{scr} to occur properly, beating the purpose of the catalyst and as a consequence, undesired \nox emissions.

To monitor gasses concentrations, chemical sensors are deployed, one of which is the \acrfull{sicfet}. The identification and quantification of gasses is normally achieved through multiple sensor in so called sensor arrays. Ideally each sensor in the array needs to have different responses to different compounds \cite{Bastuck_2019}. The deployment of multiple sensors, on the other hand, proves itself cumbersome due to the increased chances of failure, and decalibration of the system should one or multiple sensors be replaced \cite{Bastuck_2019}.

One solution to this problem is the cycled operation of one single sensor, referred as virtual multi-sensor \cite{Bastuck_2019}. By cycling the working point parameters of the sensor, different substances react differently in the sensor surface, which in turn produces different responses. \acrfull{tco}, \acrfull{gbco}, and the combination of the two have been proven to increase selectivity of \acrshort{sicfet} sensors \cite{Bastuck_2019}.

\acrshort{tco}, in contrast with a constant temperature evaluation, produces unique transient sensor responses, i.e. each gas mixture yields a slightly different sensor output. This unique gas signature increases selectivity \cite{bur2014}. Additionally, the high temperatures reached in these cycles help in the cleansing of the sensor surface, preparing it for the new mixtures to come.

Frequency modulation tries to achieve the same goal: avoid steady state responses in exchange of unique signatures that could help identify/quantify the gasses at hand. It consists on operating the sensor in \acrfull{ac}. One then can regulate the frequency of this operation and create cycles of different frequencies, similar to what is done in \acrshort{tco}. This is equivalent to \acrshort{gbco}, but with more frequency changes and achieving overall higher frequencies.

The main question is: given these set of unique sensor responses, how one can quantify the gasses that produced them? The answer lies in multivariate regression techniques. \acrfull{plsr} has been used in chemometrics extensively and it has been proven to be good at this task \cite{Bastuck_2019} \cite{wold2011}. Other multivariate regression methods, naturally, can also be used. This is the aim of this thesis work, which is shown in the following section.

\section{Aim}
\label{sec:aim}

The aim of this thesis is to investigate different regression methods, namely: \acrshort{plsr}, Ridge Regression and \textcolor{red}{(neural nets XXXX - TENTATIVE)}, and their fit to correctly quantify gas mixtures such \nox and Ammonia subjected to sensor frequency modulation.

\section{Research questions}
\label{sec:research-questions}

\begin{enumerate}
	\item Is it possible to achieve acceptable prediction levels for \nox and Ammonia using frequency modulation?
	\item Which method yields best predictions of gas concentrations?
\end{enumerate}

%\nocite{scigen}
%We have included Paper \ref{art:scigen}

%%%%%%%%%%%%%%%%%%%%%%%%%%%%%%%%%%%%%%%%%%%%%%%%%%%%%%%%%%%%%%%%%%%%%%
%%% Intro.tex ends here


%%% Local Variables: 
%%% mode: latex
%%% TeX-master: "demothesis"
%%% End: 
