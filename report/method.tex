\chapter{Methods}
\label{cha:methods}

This work's main question is: given sensor responses, how one can quantify the gases that produced them? Multivariate regression techniques have been shown to be successful, namely \acrfull{plsr} has been used in chemometrics extensively and it has been proven to be good at this task \parencite{Bastuck_2019} \parencite{wold2011}.

For example, \textcite{BUR2015225} uses \acrshort{tco} of \acrshort{sicfet} sensors alongside \acrshort{plsr} to quantify naphthalene sufficiently enough to monitor its concentration for indoor air quality monitoring. Additionally, \textcite{BASTUCK2016263} shows that, also using \acrshort{tco}, \acrshort{plsr} can  quantify ethanol and naphthalene mixtures down to the \acrfull{ppb} level.

All code was done in Python, namely in Jupyter notebooks, both for it's simplicity and for its easiness in code and data exploration. In general, the use of Python's library Scikit-Learn and its \texttt{pipeline} class alongside linear models made analysis straightforward. Additionally, Scikit-Learn's \texttt{GridSearchCV} allowed for a faster evaluation of different hyperparameters such as the number of components and shrinkage factors. 

Throughout all methods, training and test sets were split using Python's \texttt{train\_test\_split} function. 80\% of observations were assigned to the training set and the remaining 20\% to the test set. A fixed random seed of 42 was set to ensure reproducibility of results. All cross-validation evaluations were done with 5 folds.

Before any model implementation, however, the data needed to be pre-processed as described in Section~\ref{sec:preprocessing}. This was done by "transposing" the row-based, per sample  measurements into column-based, per exposure features. This reduces the number of row significantly: from 360.000 to 1500. The features were named according to Figure~\ref{fig:feat-naming}.

The evaluation and comparison of different models was made via the actual vs. predicted  plot. In it, it is possible to qualitatively see how good the predictions are. Additionally, fit metrics such as $\text{R}^2$ and \acrshort{rmse} are shown as a quantitative means of comparison.

This methodology was carried out using data from both sensors: 1 and 2. Moreover, two variations of the data from these sensors were used: first using 1500 exposures and all 480 features. The second variation was done by averaging all exposures into unique mixtures as described in Figure~\ref{fig:averaging-process} and only using average features, as described in Chapter~\ref{cha:data}.

Although all models are linear, attempts at introducing polynomial regression terms were made. However, results from these attempts were similar to the linear fit. On top of that, the addition of polynomial degrees in cross-validation grid search made computations extremely slow. Given the computational and time constraints at hand, this approach was discarded.

For further details regarding coding and implementation, the reader is referred to this work's \href{https://github.com/cosmourao/thesis}{\textcolor{blue}{repository}}, where all notebooks can be found.

\section{\acrlong{ols}}
\label{sec:met-ols}

Here treated as a baseline, \acrshort{ols} is fit using all 480 features and evaluated using the test set. 

\section{\acrlong{pcr}}
\label{sec:met-pcr}

First, a \acrshort{pca} with two \acrshort{pc}s was performed. With that, cumulative variance and score plots were made in an attempt of better visualizing and understanding the data.

Following that, a linear regression on the \acrshort{pc}s was made. The number of components was set to be between 1 and 200, and the ideal number of components was chosen via cross-validation with \acrshort{rmse} as the scoring function. Just as before, the ideal model was evaluated in a held-out test set, and an actual-vs-predicted plot was constructed.

\section{\acrlong{plsr}}
\label{sec:met-plsr}

Here, a similar procedure to Section~\ref{sec:met-pcr} was conducted. Initially, two \acrshort{pls} components were extracted, and some informative plots were made: cumulative explained variance and score plots in an attempt to better visualize the data. Here, once more, the grid of the number of \acrshort{pls} components was set between 1 and 200. The regression model was trained with the ideal number of components given by \acrshort{cv} and later evaluated in the held-out test set.

\section{Ridge Regression}
\label{sec:met-ridge}

For the shrinkage factor $\phi$, a logarithmic grid of 1000 values of  $\phi$, ranging from $10^{-10} $ to $10^4$ was set. This deliberate choice of  unusually high values of $\phi$ is further explained in Chapter~\ref{cha:discussion}. Regardless of that, \acrshort{cv} was used to find the best fit and that was evaluated in the held-out test set.